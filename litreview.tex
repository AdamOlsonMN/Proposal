\documentclass[12pt]{article}
\usepackage[utf8]{inputenc}
\usepackage{booktabs}
\usepackage{multirow}
\usepackage{rotating}
\usepackage{bigstrut}
\usepackage{tabularx}
\usepackage{prettyref}

%% fonts
\usepackage[utopia]{mathdesign}
\usepackage[scaled=.95]{inconsolata}

%% page margins, inter-paragraph space and no chapters
\usepackage[margin=1.1in]{geometry}
\setlength{\parskip}{0.5em}
\renewcommand{\thesection}{\arabic{section}}

%% bibliography
%\usepackage[american]{babel}
%\usepackage{csquotes}
%\usepackage[style=apa,natbib=true,backend=biber]{biblatex}
%\DeclareLanguageMapping{american}{american-apa}
%\addbibresource{EITC.bib}

%% For actual bib
\usepackage{natbib}
\bibpunct{(}{)}{,}{author-year}{}{,}

%% Fuck with the title
\makeatletter
\renewcommand{\maketitle}{\bgroup\setlength{\parindent}{0pt}
\begin{flushleft}
  \textbf{\@title}

  \@author
\end{flushleft}\egroup
}
\makeatother

%% for memisc
\usepackage{booktabs}
\usepackage{dcolumn}

%% define a dark blue
\usepackage{color}
\definecolor{darkblue}{rgb}{0,0,.5}

%% hyperlinks to references
\usepackage{hyperref}
\hypersetup{colorlinks=true, linkcolor=darkblue, citecolor=darkblue, filecolor=darkblue, urlcolor=darkblue}

\author{Adam Olson\\University of Minnesota\\ \today}
\title{Dissertation Proposal Draft}
\date{February 9, 2014}

\begin{document}
\maketitle

\section{Introduction}
Most scholars who study comparative welfare states refer to the United States as exceptional because they believe any welfare state the United States had was less generous, developed later, and was smaller than welfare states in comparable developed countries (See especially \citealt{andersen1990}). In a very well received challenge to this conventional wisdom, \citet{hacker2002} pointed out, the ``welfare state'' is a contested idea both among political activists and scholars and that not every country provides social policy in a uniform manner. Since then, scholars including Jacob Hacker, Christopher Howard, and others have made a compelling case that what ``is exceptional about the American welfare state is not the level of spending but the source'' of spending \citep[pg. 7]{hacker2002}.

To that point, social policy in the the United States generally belongs to one of two types: visible welfare programs and hidden welfare programs. Visible welfare programs are what most people view as welfare programs in the United States such as Medicare or Social Security. These are programs that are generally well known to politicians, interest groups, and the mass public and they are consistently salient in the national agenda. Traditionally, these are the programs which scholars refer to when they discuss measurements such as welfare effort or welfare spending - they are apparent attempts by the government to alleviate some socially undesirable problem such as poverty or hunger.

Hidden welfare programs, also known as the hidden welfare state, is social policy that is distributes benefits through the tax code as tax expenditures \citep{howard1997}\footnote{A more legalistic definition of tax expenditures is provided in the the Congressional Budget and Impoundment Control Act of 1974 as ``revenue losses attributable to provisions of the Federal tax laws which allow a special exclusion, exemption, or deduction from gross income or which provide a special credit, a preferential rate of tax, or a deferral of tax liability.''}. One prominent example of a hidden welfare program is the Home Mortgage Interest Deduction which was created in 1913 to encourage home ownership which cost the federal government around 100 billion dollars in 2010. This type of welfare through the tax code cost the federal government 1.1 trillion dollars in 2013 which by some estimates is more than was spent on visible welfare programs \citep{omb2013}. Whereas, elderly Americans can apply for social security once they turn 62, then receive a check each month, the hidden welfare state is largely administered through the tax code without much bureaucratic overhead like with Social Security. Like in the visible welfare state, the government tries to alleviate social ills such as poverty or hunger, but it does so in a way that is not evident to most people and generally uses less obvious policy making processes \citep{mettler2011}. 

While our understanding of the hidden welfare state and thus the political aspects of American social policy has increased overall due to the recent work on the hidden welfare state, there remain three sizable gaps in the literature. First, previous literature has studied hidden welfare programs largely in isolation from visible programs. This necessarily produces an incomplete story in trying to understand the politics of American social policy on a broader level. Both visible and hidden programs seek to alleviate some sort of social ill and while the programs utilize different strategies to accomplish this goal, they remain linked together by this common goal. For example, the United States has several different policies which attempt to provide easier access to housing. Home owners can deduct large parts of their mortgage interest from their federal income tax, the government also provides vouchers to help subsidize rent, and in some cases even provide housing itself. How did America develop a housing policy which incorporates both hidden and visible spending? By examining both hidden and visible programs that have similar goals we can understand how the politics of hidden policies are both similar and different from visible programs in such a way that increases our understanding of American social policy creation and development.

The second gap deals with the idea that policymaking is an iterative process that involves both a creation component \emph{and also} a persistence component. After a policy is created, it has the potential to be expanded or retrenched by enterprising members of Congress (MCs) or bureaucrats. A policy that was created fifty years ago may look nothing like it looks today because of subsequent reforms. This overlaps with the first problem of not considering visible and hidden policies together as a visible policy may be retrenched even while a very similar hidden policy is expanded. Policy affect politics and the tools used to create policy have co-evolved programmatically and politically \citep{schattschneider1960, skocpol1995}.  Historically there has been attention paid to an individual policy's entire life, including after initial passage (see especially \cite{derthick1979, hacker2002}), but most of the theories seeking to explain how the United States developed the welfare state that it did, are only concerned with initial policy creation. 

The third major gap in this type of policy making literature is its inattention to the role of congressional behavior. Research on the American welfare state is strangely devoid of insight from the voluminous literature on congressional policy making. We have tremendous insight as to the role congressional parties, divided government, congressional committees, and many other facets of congressional life have on the behavior of individual MCs and Congress as a whole. While Congress often plays a staring role in scholarly accounts of social policy creation, this research seldom formally incorporates the scholarly insight produced by congressional scholars. By applying some of the specific hypotheses outlined by congressional scholars to the historical creation and subsequent development of the American welfare state, we will be able to see how different congressional politics create different policy outcomes.

Broadly, this dissertation asks how the United States came to rely on a mix of visible and hidden policies to provide social policy. In pursuing that question, I will draw heavily on the congressional institutions and behavior literature in formulating casual hypotheses. Additionally, it will answer why a legislator chooses to use one form of welfare policy over another to advance their goals. Consequently, this dissertation will be able to examine how individual programs are created \emph{and} persist vis-a-vie each other. The juxtaposition of the traditional American welfare state literature, hidden welfare state literature, and also the congressional literature will drastically improve our understanding of how policy is created and changed  based on changing structural constraints.


%We are beginning to understand how durable traditional policies are under different conditions but do not have a deep understanding of hidden policy durability. (Cite from that patashnik article) 

\section{Literature Review}

The American welfare state literature is rich and well developed, not only in descriptive accounts but also concerning relevant policy making factors. Broadly in this section, I discuss both of those parts. In the first subsection, I overview descriptively the American welfare state, focusing on the types of the mechanisms the government uses to provide social policy. In the second subsection, I describe the traditional hypotheses of social policy making in the United States and then discuss several shortcomings of these hypotheses. 

\subsection{What Does American Social Policy Look Like?}
Many scholars have argued that social policy in the United States lags behind other developed democracies, but it is not lack of social spending that makes America unique, it is the way in which it provides social policy \citep{hacker2002}. Since the end of the second world war, the United States has predominantly used four mechanisms with which to provide social policy: visible spending, regulation, guaranteed rights, and tax expenditures \citep{pierson2007}. All of those mechanisms, but especially tax expenditures have seen explosive growth in the post war period. Tax expenditures are provisions in the tax code that favor or incentive  certain desired behavior and they are functionally equivalent to traditional government outlays or spending. In the United States tax code, there are tax expenditures for everything from buying cars and houses, to having children or working. Once tax expenditures are included in the traditional welfare effort measurement, the United States becomes average in terms of welfare spending when compared to other more `generous' nations \citep[Ch. 1]{howard2008}.

When considering the four described mechanisms of policy provision in the United States, it is important to note that these tools and their resulting policies were not created and are not maintained in isolation from one another. In fact, these tools and policies can work together, exist in conflict with one another, and layer on top of each other. Social policy in the United States seeks to alleviate a social problem. Politicians, based on the political context of the time, will utilize different tools and strategies in an attempt to pass legislation that solves that problem. Of course, political context and perceived political problems are dynamic processes and they change over time -- resulting in a disparate set of policies that seek to solve the same goal, but use functionally different tools.
\begin{figure}[h!]
  \centering
\includegraphics[width=0.9\textwidth]{graph.png}
\end{figure}

Even though tax expenditure spending should be viewed analogously to visible government spending, the politics between the two types of policy are often times quite different and reactive to one another in a way not previously outlined in the literature. One illustrative example is the case of poverty in America. Building on a state run program to aid single mothers, Congress passed a program called Aid to Dependent Children (ADC) which subsidized those state programs for that constituency. By the 1960s, through a series of eligibility changes enacted over the previous thirty years, ADC was the biggest welfare program in the country and served many more types of constituencies than just single mothers. As more Americans became eligible and started receiving ADC, it became increasingly unpopular among the public and elites. Even though the program had never been popular among Americans, it was especially vulnerable to retrenchment in the early years of the Nixon presidency. With the main poverty alleviation program unlikely to be expanded, but with the problem of poverty remaining, elites still wished to legislate around the issue. In 1974 and in spite of the desperately unpopular AFDC\footnote{The program was renamed from Aid to Dependent Children to Aid to Families with Dependent Children during the Great Society.}, Congress quietly passed a refundable tax credit for the working poor. The Earned Income Tax Credit (EITC) as it was called was a wage subsidy for low income people that would zero out their income tax liability and if any of the reward remained, the federal government would send the recipient a check for the remainder. While ADC was never expanded again and ultimately repealed and replaced with a far different program, the Earned Income Credit, social policy through the tax code, was expanded both in eligibility and generosity several times in subsequent decades \citep{stewart1991}. This is clearly seen in the above graph where the EITC explodes in cost relative to AFDC and AFDC remains basically stagnant once the EITC is introduced. The relationship between hidden and visible programs has not been  explicitly discussed in the literature yet we have much to learn from this sort of policy relationship. Why did legislators abandon expansions of ``visible'' welfare spending? Why was there a steady uptick in hidden spending even though it did almost the same thing as the now stagnant visible spending? Ultimately, we do not know why these seemingly similar programs have had such different historical trajectories with AFDC being abolished in 1995 when the EITC, which cost nearly twice as much as AFDC, was left totally untouched and is still used today. \citep{myles1997}.

In any case, American social policy uses four different types of mechanisms that both work together and compete with one another to provide services. Over time, these services and policies layer upon one another because political contexts change, which open or close different policy streams, changing the relative effectiveness and impact of the tools used to create policy \citep{kingdon2011}. Sometimes policies are retrenched and removed like in the case of ADC and sometimes the policies surrounding some issue grow and grow because other avenues are closed off like in the case of the EITC. Additionally, that different tools create different types of policies (even though they may serve the same goal), creates different politics. As illustrated above, the politics of visible programs may be radically different from the politics of tax expenditure programs. All of these factors come together to form the `exceptional' American welfare state -- a disparate web of interconnected and sometimes competing policies which have layered over time and have been enacted through one of four distinct policy making mechanisms.

With that in mind, this dissertation deals solely with the spending part of the equation, that is visible and hidden spending. The hidden welfare state literature has not been sufficiently linked with the extremely well developed research on the visible welfare state and this dissertation seeks to construct such a link. Ultimately, the conventional wisdom as it pertains to the hidden welfare state deals almost entirely with hidden \emph{spending} and analyzing hidden spending vis-a-vie visible spending drastically increases our knowledge over the spending part the American welfare regime. In any case, future research will need to keep broadening the scope while discussing explanations for the exceptional American welfare state to include rights and regulation and while they are important in general, they are not important to this story.

\subsection{Why Does American Social Policy Look This Way?}
Having briefly described the mechanisms that the government uses to produce American social policy, we now turn to an overview of why and how the United States created and maintains this type of welfare regime. The United States has a fragmented welfare state with several different types of policies for each goal it tries to alleviate. How did America reach that point? Most research on the American welfare state deals with hypotheses for why the United States has limited \emph{visible} social spending when compared to other developed countries while at the same time, largely overlooking non visible aspects of the welfare system. 

While imperfect, the explanations for the lack of visible social spending in the United States are a good start for discussing policy making overall. Political science does not understand very well what factors lead politicians to utilize one policy making tool over another in either the creation process or the maintenance process. In other words, the conventional theories are static in outlook and are unable to explain why one type of welfare spending increases while another decreases. Questions of visible spending are very much the conventional questions in American welfare state research. Pragmatically, by initially focusing on the conventional wisdom we can clearly draw out areas that are overlooked.

Some scholars think that the United States lacks visible spending because American elites and citizens do not want to have a `generous' welfare state \citep{king1973}. The scholars who make this sort of argument generally rely on terms such as political culture or public opinion data to argue that Americans do not believe in a Scandinavian style welfare system. As such, politicians do not feel the pressure to pass that style of welfare and supporting such programs might even be an overall negative for the politician. It is worth noting that support for current welfare programs is relatively high suggesting that American political culture is not a complete explanation for the relative lack of visible welfare spending \citep[Ch. 6]{howard2008}.

Other scholars argue that the relationship between labor, business and the state is responsible for variation in social spending \citep{korpi1980, swenson2004}. This argument, often times called the social democratic model or power resources theory, highlights how the interactions between capital, labor, and the state affected the welfare policy creation process. However, these hypotheses are not in full agreement as to the role of any of these actors. Some scholars emphasize that organized labor had to overcome `naturally hostile' business interests in order to expand welfare programs and that American labor was simply unable to do that. Others argue that large business interests worked with the federal government to create certain welfare programs to offload business expenses to the government. In any case, these scholars emphasize the role of business and unions (or lack thereof) to explain the creation (or lack thereof) of the American welfare state. 

The third major hypothesis is that the American institutional arrangements create a barrier to passing legislation in general and this drastically reduces the federal government's ability to create expansive welfare policies \citep{pierson1995, robertson2011}. One exceptionally American trait is that the United States has an extremely large number of spots along the policy making process where a proposal's journey can be derailed if a key person or group does not support the proposal. The multitude of veto points or veto players, as they are called, allows a large number of ways to stop legislation unless veto players are appeased in some way. This includes formal legislative points such as passing a proposal out of committee or overcoming a filibuster as well as less formal elements like interest group support. One prominent way which veto points limited visible spending in the United States is that southern Democrats delayed and reduced welfare spending during the New Deal period in order to ensure African Americans were not able to receive the same level of benefits as whites \citep{katznelson2013}. 

These three schools of thought, as I have said, only deal with visible spending as a `dependent variable.' Obviously one major problem with this approach is that it does not deal with hidden spending in any meaningful way. Further, social policy creation is occurring in a more incremental way in the United States in the less visible policy arenas. While these canonical theories may be able to explain the New Deal social legislation or the Great Society social legislation they are unable to explain how hidden policies gradually advanced over time. Additionally, they are unable to explain why the legislature adopted a hidden policy over a visible policy (or vice versa). Just within the spending category, a legislator may choose to advance his or her policy goals with either the more traditional visible spending or spending through tax expenditures.

One good way to deal with this problem is to view social policy as goal oriented rather than focusing on individual policies. The United States encourages housing in several ways, such as, by allowing home owners to deduct part of their mortgage from their income taxes, providing public housing, providing section eight housing vouchers, and several other policies. By focusing solely on one of these policies (or one type of policy) to the exclusion of the others, we cannot say that we understand how social policy is made in the United States. It is only through dealing with, in this case, housing policy as a whole that we will be able to better understand social policy making.

Another separate problem is that these theories are largely concerned with policy creation and not subsequent policy development. While policy creation is important, there are two constituent parts of American social policy -- creation and maintenance -- which need to be discussed in attempting to understand the policy making process. The lack of attention to what happens to policy \emph{after} enactment undermines our understanding of the policy making process as a whole especially when policy development can be just as important as policy enactment \citep{patashnik2008}. For instance, if we only concerned ourselves with the creation of Social Security, we would still see a program that excluded large chunks of the population, was largely not a universal program, and still had relatively weak benefits. It was only through post creation development where Social Security became the program we know today \citep{derthick1979}. 

Moreover, issues of program visibility are not as clear cut as the emerging research on the hidden welfare state suggests. For instance, program visibility should be considered more continuous than the dichotomous hidden/visible typology suggests. Some policies may be more visible than a tax credit yet less visible than something like social security. Additionally, there are questions of what visibility means to different political actors. When scholars refer to hidden or visible are they speaking about elites or citizens? Lastly, program visibility is not constant and program visibility has the ability to change over time. For instance, President Clinton pushed for large increases in the Earned Income Tax Credit in the early 1990s, suggesting that the program became more visible. Conversely, the food stamp program used to use actual stamps but now uses an EBT card which is basically a debit card, suggesting that the program became more hidden. 

Why would the United States create a hidden program \emph{instead} of a visible program? Why has hidden social policy continued to grow while visible social policy has stagnated? The United States has created lots of hidden policies before, in between, and after, the explosion of visible programs in the 1930s and 1960s yet we do not know why elites utilized different legislative strategies \citep[Ch. 2]{howard2008}. We also do not know why those hidden programs have been consistently expanded over the 20th century. Additionally, many of the hidden programs created during the non big bang periods had the same goals as the visible programs created in those periods such as encouraging higher education, reducing poverty, or providing health insurance, but we consider individual programs in isolation from the others. It is an important oversight that most conventional research on American social policy does not account for different ways of providing the policy, instead focusing unilaterally on one of the four earlier mentioned mechanisms. If policy is, in a general sense, continually being made, then there is a very large hole in our understanding of American social policy creation because we cannot explain why one type of policy was adopted over another.

\section{The Question of Visible versus Hidden Spending}
With that in mind, this dissertation will argue that the creation of the hidden welfare state was largely created by innovative legislators who wished to advance some policy goal but were constrained by the intersection of institutional and political arrangements. The ebb and ebb and flow between empowering committee chairs and central leadership in congressional institutions, along with changing political tactics and preferences place constraints on tools available to enact member preferences.

The logic of this proposition is relatively straight forward. The story of the post war American Congress is one of increasingly strict rules, more powerful centralized leadership, and more polarization \citep{rohde1991, binder2003}. Additionally, the mid 1970s saw the rise of a new brand of activist conservative who was more aggressive in attacking traditional style welfare programs like AFDC and who wished to lower tax rates \citep{hacker2007}. Within a backdrop of economic stagnation, the interaction of these two forces manifested itself in an increasingly hostile environment for traditional social policy. Any potential demand for federal assistance did not dissipate just because of this hostile environment however and many legislators still wanted to advance more expansive social policies. For a select few legislators, they found that they were able to quietly get a tax expenditure passed -- often times by attaching it to a much bigger tax bill -- which then de facto increased social spending. That story applies to expansion as well as creation, but in more dramatic style. As the same forces which encouraged hidden policy making originally intensify, the same legislators and others simply want to keep advancing their social policy goals and the tax expenditure route implicit in the hidden welfare state became increasingly the path of least resistance. 

In this section, I outline some of the forces which specifically influenced the creation and expansion of the hidden welfare state. First, I outline a series of changes in the way congressional rules and behaviors changed over the last fifty years which made passing legislation more difficult and encouraged new ways of policy making. Second, I outline how the rise of modern activist conservatism and how a new style of conservative learned to use activist government for their own means. Broadly, the ability of policies to ``remake politics is contingent, conditional, and contested" \citep[Pg. 172]{patashnik2013} and this section hopes to suggest some of the congressional and ideological conditions which allow policies to change the the politics of policy making which in turn, encourage or discourage different forms of policy making.

As I have already implied, one of the most important factors in creating and the expanding the hidden welfare state is the centralization of power in chamber leadership in the House of Representatives. Democrats in the mid 1970s wished to ``make the people who held positions of power responsible to rank and file Democrats" \citep[pg. 26, quote spoken by Rep. Donald Fraser (D-MN)]{rohde1991}. In pursuing this goal, Democrats instituted several intra-party reforms which drastically increased the power of centralized leadership. They stripped many powers from committee chairs and had them run for intra-party election to retain committee chairs. Additionally, many of the powers taken from the committees were placed within central party leadership. The logic behind the internal reforms was to empower the Democratic caucus as a whole and to remove the fiefdoms of deeply entrenched committee chairs. Instead of southern Democrats with lots of seniority setting the agenda, the caucus as a whole got to choose which policies would be advanced in a given Congress. By the 1980s, however, the caucus was empowering House leadership to negotiate policy on their behalf instead of merely acting as the agent of the party's majority \citep{sinclair1983, palazzolo1992, sinclair1998}.

The effects of moving from a committee based House of Representatives to centralized leadership empowered by the party as a whole had on visible policy making were profound. Previously powerful individuals were now unable to exert their previously significant clout to advance their own policy goals. More importantly, the caucus became an active veto point on individual member policy goals when it pertains to traditional policies. If policies needed to be supported by the majority of the caucus to have any real possibility of being passed then members with goals in tension with the party faced a real challenge. Moving past the ideological issues, a legislative session has a limited amount of agenda space and time so even if a proposed policy was acceptable to the caucus it still might not be advanced because other things are more important.

These House chamber reforms were largely adopted and expanded the Republicans when they won control of both chambers in 1994. With the movement from a committee centric Congress to a speaker centric Congress, conservatives in Congress were able to use the reforms passed by liberal Democrats in the 1970s to their own advantage as well \citep{zelizer2007}. Those with an interest in social policy now faced an even more hostile set of veto points because conservatives controlled the chamber and were actively trying to cut welfare spending. In the New Deal period and prior to the 1970s reforms more generally, the decentralized nature of the House chamber and the lack of well sorted congressional parties encouraged bipartisan and ideologically diverse voting coalitions \citep{poole1997}. This allowed even minority party members to realistically pursue their policy goals. The Republicans swept into office viewed themselves as ideological purists and were largely unwilling to compromise with the now minority Democrats \citep{hacker2006}. Combined with the largely centralized chamber power, Republicans were able to simultaneously pass large parts of the Contract with America and also discourage legislators who may have wanted to temper any welfare cuts \citep{aldrich2000}. 

In the Senate, the story is very much the same except it was the reciprocal norm of respect eroding \emph{alongside} an empowerment of chamber leadership \citep{sinclair1986}. Since the early 1980s individual senators have spent more resources empowering their party's leadership and those leaders changing the Senate agenda to maximize inter-party cleavages while minimizing their own party's ideological differences \citep{lee2008}. An obvious consequence of this type of agenda control is that it encouraged ideological sorting among Senators to the point where all conservatives were in one party and all liberals in the other. It was already difficult to pass legislation through the Senate due to supermajority requirements, so as parties became more homogenized the cost of moving the status quo increased \citep{koger2010}. Like in the House, a costly legislative process encourages members to develop new ways to legislative -- namely, through the tax code.

Outside of individual chambers of Congress, divided government also makes legislating difficult. Prominently, \cite{mayhew1990} argues that divided government has no real effect on `major' legislation passed in a given Congress. In this case, major legislation means highly salient legislation and by definition does not account for hidden spending. Conversely, divided government has been mentioned as relevant to the creation of hidden welfare spending. \citet[Ch. 4]{howard2008} argues that divided government is associated with more hidden welfare policies being made as it is harder to pass visible legislation in a divided government situation. Since 1970, the United States has had unified government 30\% of the time compared to 70\% for the earlier part of the 20th century. Again, as it becomes harder to pass legislation legislators need to find more innovative ways to advance their policy goals.

These are big changes in the American Congress which, I propose, go hand in hand with creation and expansion of the hidden welfare state at the expense of the visible welfare state. Over the last forty years, legislation has become increasingly more difficult to pass through Congress. Through a rise of restrictive rules, centralized chamber leadership, increased obstruction, and increased polarization, passing traditional social policy through the traditional policy making process became too costly for politicians. Legislators who still wanted to advance social policy needed to find other ways to achieve their goals. By making policy through the tax code, legislators found a tool that did not require as much political capital to advance and was not even on the radar of most other members of Congress. By using the tax code, legislators constructed the hidden welfare state by carving out exemptions and constructing expenditures which allowed them to reward certain behavior without most other MCs realizing what was happening. 

As Congress was changing, so to was American conservatism. The rise of the modern American conservative movement came directly as a reaction to the New Deal and the Great Society, and the activist state more generally \citep{critchlow2007, zelizer2010}. They resented taxation in general and using the money to fund a welfare state was deeply unpopular among conservatives ascribing to New Deal based conflicts of the time. In post war America, the overall conflict between liberals and conservatives was over the scope and size of the federal government. Broadly speaking, conservatives wished to retrench New Deal and Great Society programs and generally wanted the federal government to do less but most did not want to remove the safety net altogether. 

The nature and tactics of conservatives started to change in late 1960s though with an increase of conservatives wishing to use the government to pursue their own policy goals instead of just stopping activist government \citep{teles2007, skocpol2007}. Put differently, some conservatives began operating beyond the ``New Deal Consciousness" of post war America. To that end, conservatives, began to use the levers of centralized government created by liberals in the 1970s to pursue their own policy goals \citep{hacker2004}. This new style of conservatism exacerbated the growth of the hidden welfare state as many of the newer activist style conservatives learned they could advance a policy goal and also say they were cutting taxes. 

Once Republicans took over Congress in the 1995, they inherited an already centralized leadership structure and Speaker Gingrich proceeded to make it even more centralized \citep{roberts2003}. Gingrich changed several House rules to enact preferred policy preferences from the Contract with America. Additionally, the further increase in leadership centralization allowed for further polarization and an increase in House based veto points. With key leadership positions in the House of Representatives controlled by the `new' brand of conservatives and with the increases in ideological homogeneity, negative agenda control would totally stymie visible policy creation. As with other changes described thus far, lawmakers who wished to advance social policies needed to take a different strategy. 

The evolution and rise of modern conservatism combined with the changes of congressional behavior and rules, I propose, can largely explain the creation and expansion of the hidden welfare state. As Congress changed so did the policy it produced. Similarly, as the ideological and tactical inputs of conservatives changed, so did policy. Consequently, politicians deliberately choose less visible policy making techniques because it is easier for those policies to pass and also become entrenched. Like with the visible welfare state, related policies layer on one another, resulting in the fragmented welfare state that we have. Hidden and visible policies work in tandem in providing the American social safety net. More generally, public policy is created over time and is very tightly linked with the institutional arrangements that approve, develop, and administer the program \citep{pierson2004b}. The types of policies that are passed depend on the institutional constraints that the political actors are behaving under and the policy preferences of the actors themselves. By examining the rules and actors involved with a political process -- in this case the creation and development of the American welfare state -- we can illuminate the policy making process, American public policy, and the way political actors and political institutions interact.

\subsection{Expectations}
Generally, this dissertation expects that when many veto points are engaged, Congress will pass more hidden policies and when fewer veto points are engaged, more traditional visible policies will be created. I expect policy durability follows a similar pattern. Visible policy, once expanded, becomes less subject to retrenchment over time while visible spending, almost immediately will persist very easily. There are obvious examples of veto points on many levels of scale, such as divided government or a hostile committee chair wielding agenda control. 

Even with this general pattern of expectations, there are several more specific expectations within that framework. First, I expect that hidden policies are created and expanded by politically powerful, senior, and electorally safe MCs. The reasoning for this is that I expect to find many hidden programs are created or advanced by last minute amendments or additions to committee reports or floor bills. It would be much easier for people who chair the relevant committee or have seniority in general to attach such an amendment without drawing scrutiny from other members. Additionally, it seems less likely that an MC who advances hidden programs would expect significant electoral benefits from claiming credit for the program so I expect MCs who desire reelection as their primary goal to focus on more visible policies. \footnote{
The idea that MCs pursue their goals differently and use policy to achieve reelection is not new \citep{fenno1973, kernell1999}. Regardless, I assume that a legislator that depends on policy advances for election would focus on very high profile legislation.} 

A second specific expectation is that I expect policy creation to vary in conjunction with the conditions outlined by conditional party government (CPG) \citep{rohde1991}. I expect that when the `condition' is high -- that is when the two parties are internally ideologically homogeneous and externally polarized from one another -- hidden policy will flourish while visible spending will become harder to pass and expand. Conversely, when the condition is low, that is the parties are ideologically heterogeneous internally and have overlap externally, it will be much easier to pass  and expand visible policy. One of the reasons for this is that more MCs are empowered during periods of low CPG so they are better able to enact their policy preferences. There is no need to hide policy in order to pass it. During high periods of CPG, the reverse is true as fewer MCs are empowered and policy driven members need to think of ways to overcome the consolidated veto points.

Third, while there is disagreement of the role divided government played in passing `major legislation,' I expect that divided government of all types will encourage hidden policy making, while more unified government will encourage visible spending. In times of divided government, legislators who wish to advance policy goals need to potentially overcome more disagreement over what constitutes good policy so they have to resort to hidden policy making tools. During times of unified government, the government is incentivized to pass visible policy so they can claim credit for electoral benefits.

In general, the story of changing structural constraints will guide my causal explanation for why MCs choose to advance one policy type over another. Presumably some of these expectations will interact with one another in interesting ways. For instance, President Bush passed a major increase to Medicare in 2003, a time when both houses of Congress were Republican controlled. One plausible explanation is that the combination of a high CPG condition and unified government allowed the President to convince the relatively few veto point holders to go along with the expansion \emph{in spite} of the increased prominence of `new' conservatives. In any case, these sort of situations will need to be dealt with as they arise. With that in mind, I turn to crafting a research design with enough flexibility required for such an undertaking, but enough analytic rigor to prove the argument.

\section{Research Design}
From a broad stand point, this dissertation will examine both macro trends in social policy making and maintenance along with trends in individual policies within two distinct social policy goal areas. In pursuing that strategy, this dissertation will use a combination of quantitative and qualitative methods in order to make it's case that certain Congressional institutions and trends encourage one type of social policy making over another. Additionally, this dissertation will extend over a wide swath of the 20th and 21st century enabling the ability to examine changes in social policy making against the backdrop of changes in the congressional behavior and institutions.

By starting with a quantitative examination of visible versus hidden social policy trends, the dissertation will show where changes occur in spending behavior. This sets the stage for subsequent analysis, but also continues to fill the gap in American social policy literature by showing that substantial parts of the American welfare system are distributed through the tax code. Additionally, it helps show that the growth of more visible polices are in a dynamic relationship with hidden policies where policies might change in reaction or in tandem to one another. Additionally, by pairing the larger spending analysis with two large scope case studies, we create a methodological strategy where we examine the same policies and politics through different lenses, bringing even more insight.

In designing the dissertation overall, I think a combination of both \cite{zelizer1998} and \cite{schickler2001} work together well for the sort of argument I am making. \cite{zelizer1998} examines the structural constraints on long time House of Representatives Ways and Means Chairman Wilbur Mills and how they shaped the types of policies he pursued (and also his attempts at changing the constraints). Zelizer is an academic historian so his book is largely concerned with constructing a narrative about Mills and these constraints through time. I am looking at the question of choosing which legislative tool through the same sort of lens. Some constraints encourage different behavior and politicians react accordingly. Additionally, \cite{schickler2001} tells a story about how diffuse interests come together at different points in Congressional history to change the institutional rules and constraints. His work is largely historical in outlook as well, but he shows the effect of some of those interests using traditional quantitative analyses such as regression. I anticipate that sort of strategy will be extremely helpful towards my dissertation.

With that in mind, the types of data this dissertation will use are diffuse. Beyond any primary source material, I will need to create an argument using nearly a century of history, there are a plethora of secondary statistical data as well. I will use roll call data at the committee and chamber level for bills and amendments. The Earned Income Tax Credit was passed as a relatively small amendment to a much larger tax credit. Moreover, \cite{adler2012} compiled a complete data set of every bill introduced in the United States Congress since 1947 along with characteristics about the sponsor, number of cosponsors, and how far the bill advanced. Such data could be amended to include characteristics about program visibility, and policy making tool. Additionally, if I am to examine how policy visibility changes, some sort of measurement of that is required. Congressional speech or bill introduction / co-sponsorship would serve as a good measure for how visible a program is to MCs while opinion articles in major newspapers may serve as a good proxy for how visible a program is to citizens. 

%lol this table looks super bad in code. Don't ever change it.

\begin{table}
\centering
    \begin{tabularx}{\textwidth}{XXX} \toprule
           & \textbf{Within Case Comparison} & \textbf{Cross Case Comparison                                                                              } \\ \midrule
    \textbf{Individual Policy Level} & Trace Creation and Development of Individual Policies        & Compare Creation and Development of Individual Policies against other policies within a policy area \\
    \textbf{Policy Area Level}       & Trace Creation and Development of Policies in a Policy Area as a Whole & Compare Creation and Development of Policy Areas against other Policy Areas                         \\ \bottomrule
    \end{tabularx}
  \caption{Multi-Level Case Study Design}
  \label{tab:casestudy}
\end{table}

The body of the dissertation will be based around a series of case studies involving individual policies, two separate issue areas of social policy, cash assistance and housing policy. Such a strategy has superficial benefits, such as increasing our understanding of the different vehicles which the government uses to provide benefits for a given policy area. A more methodologically rich understanding of this strategy allows a research design which combines both within-case and cross-case comparison in a unique two level way \citep{george2005, goertz2012}. As described in Table One, by treating individual policies as cases we are able to examine how institutional changes affected a given policy over time, allowing for within-case analysis. At the same time, we are able to trace individual policy creation development in relation to other policies in the same policy area, allowing for cross-case comparisons. Similarly, when we treat policy area as the unit of analysis, we can examine how a policy area changes over time and how policy areas developed in relation to one another. By using this type of case study approach, I am able to develop a very flexible broad based approach to this dissertation where we can see how policies and policy areas developed and changed in response and alongside one another. 

This collection of policies described in table two are representative of the type of spending policies that politicians choose to utilize in providing social policy. In selecting policies to analyze, one must be careful to ensure the method of policy making represents a wide swath of options available to policy makers. Considering this, I have created a three level structure, equating to three policies for each policy area representing a different level of program visibility. My classification scheme is as follows: fully `hidden' policies are tax expenditures, mixed policies are generally a form of voucher where it isn't totally obvious to outsiders you are benefiting from government social policy, and fully visible policies are actual government services. 

\begin{table}
\centering
    \begin{tabularx}{\textwidth}{XXX} \toprule
           & \textbf{Housing Policy} & \textbf{Cash Assistance Policy} \\ \midrule
    \textbf{Visible Policy} & Public Housing        & Aid to Families with Dependent Children \\
        \textbf{Mixed Policy} & Housing Choice Voucher Program        & Supplemental Nutrition Assistance Program \\
    \textbf{Hidden Policy} & Home Mortgage Interest Tax Deduction        & Earned Income Tax Credit \\ \bottomrule
    \end{tabularx}
  \caption{Types of Policies within Policy Area}
  \label{tab:types}
\end{table}

As shown in table two, I will examine policies within both housing policy and cash assistance policy. These two groups of policies often encompass what people mean when they say `welfare' and they also experienced a lot of policy variance in the 20th century. Pivoting, for visible policies I use Public Housing and AFDC as they are highly visible programs to multiple stakeholders. Politicians can easily point to welfare offices and public housing buildings because they are tangible and relatively obtrusive and also the recipients of those programs tend to have a much larger and more obvious stigma attached to them than other types of welfare programs \citep{soss2002}. Mixed policies are still efforts to provide welfare, but they are less obvious both to citizens and probably lawmakers. For instance, housing vouchers are essentially rent subsidies, but much of the time it is not evident that beneficiaries are getting government assistance. SNAP (Supplemental Nutrition Assistance Program) works in a similar way where beneficiaries receive an EBT card which acts like a debit or credit card and thus is quite unobtrusive. Lastly, hidden policies are well defined in the literature but they are most often some sort of distortion in the tax code to produce some outcome. In the case of the two policies I chose, one policy encourages work for low income people by acting as a wage subsidy and the other encourages people to buy homes. As I referenced earlier, policy visibility changes over time, but this listing exists as a general framework moving forward on the research. I fully anticipate policies changing categories and that political actors will try to change policy visibility. While it seems likely that politicians change policy making strategies to more easily accomplish their goals, it seems equally plausible that politicians will try to change visibility to suit their goals.

One benefit to this sort of analysis is the ability to change the concept of political time, allowing it to be more than a story about how `great people' accomplish their goals. Often historians who wish to ``change political time" mean they try to incorporate contextual information which involves their quantity of interest which changes over time \citep{zelizer2002}. They include social movements, wars, realigning elections, etc. This dissertation will include contextual information of a legislative nature. Coalitions shift, program visibility shift, institutions shift. There are formal institutions which are relatively easy to measure which constrain and encourage program creation and visibility. This allows us to make predictions concerning which types of policies are created and how they develop based on structures rather than individuals. 

One drawback to this style of research design is that it relies on a collection of data sources, narratives, and interpretation to develop the argument. While in many ways it may be a strength to build a case upon a collection of sources, such a strategy lacks scientific rigor in the traditional sense \citep{king1994}. Regardless of these claims, it is more difficult to find causal links with this sort of research design than it might be with a less ambitious set of arguments, thus raising the bar for me. Hopefully this increased difficulty produces a better document overall which persuades more people because of the methodological pluralism contained within.

Overall, the research design for this project is flexible enough to account for many covarying parts while remaining powerful enough to tell a causal story. By combining a multitude of primary, secondary, quantitative and qualitative data sources I will be able to weave a strong multi-methodology dissertation which does not rely too heavily on one particular analysis to make an argument. Additionally, by using several different units of analysis, this dissertation will be strongly positioned to isolate relevant causal factors in the pursuit of identifying reasons why MCs choose to advance one form of policy over another. An additional benefit to that form of research design is that it strongly positions the dissertation to tell a readable and compelling narrative to underlay the analyses in the document. 

\section{Chapter Outline?}


\newpage
    \bibliography{EITC}{}
\bibliographystyle{jpe}

%\printbibliography
\end{document}

